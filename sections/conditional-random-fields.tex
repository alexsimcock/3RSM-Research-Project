\documentclass[../main.tex]{subfiles}

\begin{document}

The conditional random field (CRF) \autocite{lafferty-crf-2001} is an alternative graphical model for structural prediction that can be understood as a generalisation of the HMM, retaining similar graphical structure but adopting more flexible assumptions.
The CRF falls into the class of \textbf{discriminative} models which, unlike the generative NB and HMM, make no assumptions of the process by which data is produced.
The individual elements of discriminative models are generally less interpretable than their generative counterparts, but the models are more powerful overall due to their potential to leverage a greater amount of context.

%------------

As in the HMM, the elements of $Y$ and $X$ are indexed by nodes of the graph, however arcs no longer indicate a process of transition or emission but can instead be seen as a two-way influence - it is for this reason that the arcs are now undirected.


This is as the new model is discriminative.
Importantly unlike the HMM probabilities are not required to be solely dependent on the current state $Y_t$ and can take influence from anywhere so the arcs may join any set of nodes.
With these relaxations many new structures become possible, allowing for a much richer context window and thus more reliable results when estimating $p(Y \mid X)$.
An example of potential CRF structure is given in figure \ref{fig:crf-diag}, demonstrating that connections are possible between all nodes in both the hidden and observed chain.
As such, a feature of a given observation may influence the next label. Or otherwise the values of all observations could influence tags.
Restrictions are placed on the number and kind of connections made by a CRF for the sake of computational/algorithmic complexity and avoiding issues of over-fitting.
Most commonly this is divided into the \textbf{linear chain CRF} and \textbf{general CRF}.

%\subfile{tikz-figures/old-crf-diagram}

\subfile{tikz-figures/lc-crf-diagram}

\subsubsection{Generative and Discriminative Models}
The HMM which has been discussed thus far is a generative model...
The CRF is discriminative not generative meaning that although $p(Y \mid X)$ is constructed $p(X)$ is NOT.
The CRF has a number of favourable properties, namely enabling a richer set of features to be incorporated into predictions and also avoiding label bias that other alternatives to the HMM such as the Maximum Entropy Markov Model (MEMM) suffered from.
It excels at labelling and parsing sequential data due to a learning approach that is able to infer relationships over many rounds of training.
It is applied often in modern machine learning systems for both natural language processing and computer vision.
The CRF is discriminative rather than generative.
Unlike the HMM, the CRF takes a discriminative (rather than generative) approach. In doing so it does not directly model the bl,l,flfe,.



\subsection{Algorithms for Inference}

When working with linear chain CRFs we are able to easily adapt the suite of iterative algorithms that we previously developed for HMMs.
As in that case the calculations still provide exact figures however the key variables $\alpha$, $\beta$, $\gamma$, $\delta$ lose the neat interpretations as themselves being probability distributions - as a result of the switch from a generative to discriminative approach.

\subsubsection{Viterbi Algorithm}
\subsection{Algorithms for Learning}
\subsubsection{Baum-Welch Algorithm}
\subsubsection{Pietra Algorithm}



\subsubsection{Label Bias Problem}

The CRF was initially motivated by the label bias problem.
Occurs when the data is not representative of every possible outcome.
Label states with only one outgoing transitions take little heed of their observation.
Regardless of how often this observation has been seen all of the probability mass is passed along.
If one overall word sequences is seen more often in training, this will now be chosen...

\subsection{Linear Chain CRFs}

\begin{definition}[Linear Chain CRF]
    Let $X$, $Y$ be random sequences/vectors variables.
    $\theta_k$ be parameter vector and $\mathcal{F} = \{ f_k (y,y',x) \}_{k=1}^K$ be a set of feature functions.
    The conditional random field is then a probability distribution.
    \begin{equation*}
        p(Y \mid X) = \frac{1}{Z(X)} \prod_{t=1}^{n} \exp \left\{ \sum_{k=1}^K \theta_k f_k (Y_t, Y_{t-1}, X) \right\}
    \end{equation*}
    Where $Z(X)$ is a normalising function over all values of $Y$:
    \begin{equation*}
        Z(X) = \sum_{\forall Y} \prod_{t=1}^{n} \exp \left\{ \sum_{k=1}^K \theta_k f_k (Y_t, Y_{t-1}, X) \right\}
    \end{equation*}
\end{definition}

Can be thought of as a HMM where the transition probability is not fixed as a scaled $p (Y_t = i \mid Y_{t-1} = j)$. We can incorporate features for example from the current observation, orrr alll observations, here actually although we have written $X$, computationally we would consider only those elements of $X$ pertinent to timestep $t$.
i.e. the assumptions are much less restrictive and relationship between $X$ and $Y$ is two-way.

As the number of features increases, computing the probabilties becomes increasingly complex.
$Z(X)$ is also exponentially LARGE, but calculable by forward-backward!

Linear chain CRFs are considered more standard for industrialised use and impose the restriction that elements in the hidden state only interact with their neighbours.

\begin{itemize}
    \item Desirable features
    \item Computing advantages of linear chain CRFs
\end{itemize}

\subsection{Generalised CRFs}

\begin{definition}[Generalised CRF]
    Let $G$ be
    \begin{equation*}
        p (Y \mid X) = \frac{1}{Z(X)} \prod_{a=1}^A \Psi_a (Y_a, X_a)
    \end{equation*}

\end{definition}

Differs from general factor graph as $Z$ is now a function of the input. Previously $Z$ may not have been computable but $Z(X)$ may be

\begin{definition}[Conditional Random Field]
    Let $G = (V,E)$ be a graph such that $Y$ is indexed by the vertices of $G$: $Y = (Y_v)_{v \in V}$. $(X,Y)$ is a \textbf{conditional random field} if, when conditioned on $X$, the random variables $Y_i$ obey the Markov property with respect to $G$:
    \begin{equation*}
        p(Y_i \mid X, Y_j, i \neq j) = p(Y_i \mid X, Y_j, j \sim i)
    \end{equation*}
    Where $j \sim i$ means $j$ and $i$ are neighbours in $G$.
    %In other words, the likelihood of one tag transitioning to another is reliant exclusively on the present state and no wider memory. IS THIS TRUE? Surely that's a linear chain
\end{definition}

The following aspects of CRFs are to be investigated and discussed in the final paper:

\subsection{Feature Incorporation}

Alongside the new features we have already been granted the CRF is able to include new features like the suffixes and position in the sequence.

\subsection{Inference from Linear Chain CRFs}

Inference in linear chain CRFs is generally parallel to that in HMMS.
Initially it is intractable and exponentially complex to calculate by brute force.
Since we also will need to call this routine often in training, it may be that approximations could be used to even further speed up algorithms - although carry risk of unpredictable effects.
We can, in the linear case, use the same algorithms as in the HMM (generalised models require more in-depth approaches).

\subsubsection{Re-denoting the HMM Inference Algorithms}
First we define the earlier algorithms with new notation.
The HMM:
\begin{equation*}
    p (Y,X) = \prod_t \Psi_t (Y_t, Y_{t-1}, X_t)
\end{equation*}
Where,
\begin{equation*}
    \Psi_t(j,i,x) \coloneqq p (Y_t=j \mid Y_{t-1}=i) p (X_t=x \mid Y_t=j)
\end{equation*}

The forward algorithm caches inner sums... there are $\mathcal{Y}$ such variables
\begin{equation*}
    \alpha_t(j) \coloneqq p (X_{1:t},Y_t=j) = \sum_{\text{all } Y_{1:t-1}} \Psi(j,Y_{t-1},X_t) \prod_{t'=1}^{t-1} \Psi_{t'}(Y_{t'},Y_{t'-1},X_{t'})
\end{equation*}
Which can be calculated by recursion
\begin{gather*}
    \alpha_1(j) = \Psi_1(j,Y_0,X_1) \\
    \alpha_t(j) = \sum_{i \in \mathcal{Y}} \Psi_t(j,i,X_t) \alpha_{t-1}(i)
\end{gather*}
Backwards is reverse...

Again allowing for calculation of $p (Y_{t-1}, Y_t \mid X)$.
For viterbi we swap out summation with maximisation.

The FB algo is now identical! Simply with a new definition of $\Psi_t$.

% WE NEED NOTATION FOR MORE THAN JUST XT (ALL RELEVANT X)
In the CRF
\begin{equation*}
    \Psi_t (Y_t,Y_{t-1},X_t) = \exp \left\{ \sum_k \theta_k f_k(Y_t,Y_{t-1},X_t)\right\}
\end{equation*}

Although the algorithm is the same our interpretation does change - since the model is discriminative and not generative. The $\alpha$ and $\beta$ variables do not directly mean anything.

Then
\begin{equation*}
    p (Y_t \mid X) = \frac{1}{Z(X)} \alpha_t(Y_t)\beta_t(Y_t)
\end{equation*}

General CRFs will require other methods such as monte carlo (unbiased) or variational (which are biased but faster and borrow from the field of optimisation).
Any graph inference techniques can be used.

\subsection{Parameter Estimation}

Provided and based off of \autocite{della-1997-crfparam} in the original thing.

\subsubsection{Adapting the Baum-Welch algorithm}

%Rates of convergence \& symbiosis with other machine learning techniques
%\section{Model Extensions}
%\subsection{Unsupervised Parameter Learning / Baum-Welch Algorithm}
%Can be used to train model parameters if there is no labelled data!!!!

\end{document}

- **Conditional random fields (CRFs)** are a probabilistic framework for modeling structured outputs, such as sequences, trees, and graphs, given a set of input features.
- **CRFs** combine the advantages of **discriminative classification** and **graphical modeling**, allowing for rich, overlapping features and compact representation of complex dependencies among output variables¹[1].
- **CRFs** can be seen as a generalization of **logistic regression** and a discriminative analogue of **hidden Markov models (HMMs)**, which are generative models of structured data²[2]³[3].
- **CRFs** can be trained by **maximum likelihood estimation** using gradient-based optimization methods, such as **L-BFGS** or **stochastic gradient descent**. The gradient computation requires **inference** in the model, which can be done exactly for some structures, such as linear chains, or approximately for others, such as general graphs, using methods such as **belief propagation** or **Markov chain Monte Carlo**.
- **CRFs** have been successfully applied to many tasks in **natural language processing**, **computer vision**, and **bioinformatics**, often achieving state-of-the-art results⁴[4]. They can also be extended to include **latent variables**, **higher-order factors**, and **non-linear features**.